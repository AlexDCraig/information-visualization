\documentclass[journal]{vgtc}                % final (journal style)

%% These few lines make a distinction between latex and pdflatex calls and they
%% bring in essential packages for graphics and font handling.
%% Note that due to the \DeclareGraphicsExtensions{} call it is no longer necessary
%% to provide the the path and extension of a graphics file:
%% \includegraphics{diamondrule} is completely sufficient.
%%
\ifpdf%                                % if we use pdflatex
  \pdfoutput=1\relax                   % create PDFs from pdfLaTeX
  \pdfcompresslevel=9                  % PDF Compression
  \pdfoptionpdfminorversion=7          % create PDF 1.7
  \ExecuteOptions{pdftex}
  \usepackage{graphicx}                % allow us to embed graphics files
  \DeclareGraphicsExtensions{.pdf,.png,.jpg,.jpeg} % for pdflatex we expect .pdf, .png, or .jpg files
\else%                                 % else we use pure latex
  \ExecuteOptions{dvips}
  \usepackage{graphicx}                % allow us to embed graphics files
  \DeclareGraphicsExtensions{.eps}     % for pure latex we expect eps files
\fi%

%% it is recomended to use ``\autoref{sec:bla}'' instead of ``Fig.~\ref{sec:bla}''
\graphicspath{{figures/}{pictures/}{images/}{./}} % where to search for the images

\usepackage{microtype}                 % use micro-typography (slightly more compact, better to read)
\PassOptionsToPackage{warn}{textcomp}  % to address font issues with \textrightarrow
\usepackage{times}                     % we use Times as the main font
\renewcommand*\ttdefault{txtt}         % a nicer typewriter font

%% declare the category of your paper, only shown in review mode
\vgtccategory{Research}

\title{Visualizations of How NBA Player Salaries Relate to Individual and Team Performance}

\author{Alex Hoffer, Austin Nguyen, Prathveer Rai}

\vgtcinsertpkg

\begin{document}

\maketitle

\section{Introduction}
The following subsections explain what the problem is we hope to elucidate through the use of information visualization, the importance of visualization as a tool in understanding the problem, potential users of our visualizations, and how we plan on approaching our visualizations in subsequent assignments.

\subsection{Problem Explanation}
In the National Basketball Association (NBA), teams are allowed 15 players on their rosters and each player is paid a salary established in their contract. Each team can spend a total of ***x dollars*** on their players, and this is referred to as their \emph{salary cap}. The salary which a player is given can be based on a number of complex factors such as an impending increase in salary cap, a demand for that player's skill set, or improvements to team chemistry, but it can generally be agreed that the amount a player is paid should correspond to their on-court performance. It is therefore an advantage for teams to get good players on cheap contracts so that they can have flexibility within the salary cap to spend money on getting more good players. An excellent example of a great player on a cheap contract is Stephen Curry, who signed a meager contract when he was injurious, and since then has improved to become a premier NBA player. On the flip side, since there is a salary cap, teams should be wary of spending a lot of money on players who do not deserve it. Some examples of players who are generally agreed to be overpaid considering their skill level are Timofey Mozgov of the Los Angeles Lakers and Bismack Biyombo of the Orlando Magic. The problem, then, is that since each team must carefully spend their money in order to maximize team success, some players end up overpaid, and this leads to teams underperforming. But what aspects of player performance can be used to assess whether or not they deserve their salary? It is the goal of this project to discover through the use of information visualization how statistical domains of player performance such as points per game, team performance in terms of their win/loss record, and player salary are related in an effort to inform teams on which players are actually worth their price tag so they may more wisely allocate their financial resources and thus improve their chances of winning.  

\subsection{Importance of Visualization}
It is easy to get lost amidst the plethora of statistics there are available to fans of the NBA. From simple statistics like points or assists per game to complex measurements such as Player Efficiency Rating (PER), those interested in evaluating a player may not get an accurate impression because of how difficult and unclear it is to read these numbers and mentally compile them in a meaningful way. Here is where visualization comes in: being able to graphically represent basketball statistics means that we can tell a story about a player's performance in relation to his salary without bogging the reader down with unnecessary details. We can facilitate their understanding by presenting the information they desire to know in a way that makes more sense to them than merely showing them a table of numbers.  

\subsection{Potential Users and User Groups}
There are a myriad of potential users of the information visualizations we propose. A sample of groups that may be interested are listed below:
\begin{itemize}
\item \emph{NBA Fans: } Fans of the NBA may want to see how the players on the team they like are contributing to their team's success or are, in fact, detrimental.
\item \emph{NBA Management: } Those who occupy management positions that are in charge of player personnel within NBA teams want to know how much money they should offer a player, and knowing how much bang for their buck they've gotten from a player will aid in this pursuit.
\item \emph{NBA Agents: } Agents of NBA players will want to know how much money they should demand for their player in the course of contract negotiation.
\end{itemize}

\subsection{General Approach to Visualization}

\section{Visualization Tasks}

\section{Data Sources}

\section{Data Organization in Terms of ER Diagrams}

\section{Relational Database Implementation of ER Diagrams}

\section{Design of the Visualization Interface}

\bibliographystyle{abbrv-doi}
\bibliography{template}

\end{document}

